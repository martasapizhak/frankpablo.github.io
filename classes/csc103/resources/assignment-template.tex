% -------------------------------------------------------------------------------
% This is all preamble stuff that you don't have to worry about.
% Head down to where it says "Start here"
% -------------------------------------------------------------------------------
 
\documentclass[12pt]{article}
 
\usepackage[margin=1in]{geometry} 
\usepackage{amsmath,amsthm,amssymb}

\usepackage[parfill]{parskip}
\parskip=\baselineskip
 
\newcommand{\N}{\mathbb{N}}
\newcommand{\Z}{\mathbb{Z}}
 
\newenvironment{exercise}[2][Exercise]{\begin{trivlist}
\item[\hskip \labelsep {\bfseries #1}\hskip \labelsep {\bfseries #2.}]}{\end{trivlist}}

\newenvironment{solution}[1][Solution]{\begin{trivlist}
\item[\hskip \labelsep {\bfseries #1}\hskip \labelsep {\bfseries}]}{\end{trivlist}}

\usepackage{fancyhdr}
\pagestyle{fancy}
\lhead{Submitted by: \studentName\\
\collaborators}
\rhead{CSC250 Spring 2021 - Assignment \assignmentnum\\
\today{}}
\cfoot{p. \thepage}
\renewcommand{\headrulewidth}{0.4pt}
\renewcommand{\footrulewidth}{0.4pt}
 
\begin{document}
 
% --------------------------------------------------------------
%                             Start here
% --------------------------------------------------------------

\newcommand{\studentName}{YOUR NAME HERE}

\newcommand{\assignmentnum}{X} % Replace X with the appropriate number

\newcommand{\collaborators}{
	% Comment out the line below if you worked alone
	with \textit{COLLABORATORS' NAMES HERE}
	% Uncomment the line below if you worked alone
	% \textit{I did not collaborate with anyone on this assignment.}
}

% -----------------
% Problem xyz
% -----------------
\begin{exercise}{xyz}
Delete this text and write the problem statement here.
\end{exercise}
 
\begin{solution}
Your solution here. 

To cite any resources you used in solving the problem, include a \texttt{\textbackslash{}bibitem\{...\}} entry in the \textbf{References} section, and then call the \texttt{\textbackslash{}cite\{...\}} command like this~\cite{sipser}.
\end{solution}

% -----------------
% Problem abc
% -----------------
% Use \clearpage to put each subsequent problem on a new page
\clearpage

\begin{exercise}{abc}
Delete this text and write the problem statement here.
\end{exercise}
 
\begin{proof}
You can also use the \texttt{proof} environment instead of \texttt{solution}, which will give you a nice little box at the end of your argument.
\end{proof}

% -----------------
% References
% -----------------
\clearpage

\begin{thebibliography}{9}
\bibitem{sipser} 
Sipser, Michael. 
\textit{Introduction to the Theory of Computation.} 
Course Technology, 2005. ISBN: 9780534950972. 
 
\bibitem{knuthwebsite} 
Knuth: Computers and Typesetting,
\\\texttt{http://www-cs-faculty.stanford.edu/\~{}uno/abcde.html}
\end{thebibliography}

% --------------------------------------------------------------
%     You don't have to mess with anything below this line.
% --------------------------------------------------------------
 
\end{document}